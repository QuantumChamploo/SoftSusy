\documentclass[12pt]{article}
\usepackage{epsfig}
\usepackage{psfrag}
\usepackage{enumitem}
\usepackage{latexsym}
\usepackage{indentfirst}
\usepackage{fancyhdr}
\usepackage{amssymb}
\usepackage{amsmath}
\usepackage{amsfonts}
\usepackage{pifont}
\usepackage{cite}
\usepackage{bbold}
\usepackage{ulem}
\usepackage{color}
\usepackage[footnotesize]{caption2}
\usepackage{graphicx}
\usepackage[center,footnotesize,hang]{subfigure}
\usepackage{url}
\usepackage[dvipsnames]{xcolor}
\usepackage{array}
\usepackage{slashed}


\newcommand {\red}{\color{red}}

\textwidth 16.5 cm

\textheight 25 cm \topmargin -2.5 cm \hoffset -1.5 cm

%
%%%%%%%%%%%%%%%%%%%%%%%       FRONTPAGE       %%%%%%%%%%%%%%%%%%%%%
%

\begin{document}

\section*{Notes on Model B}

These notes describe an investigation into the flavored gauge mediation scenario in {\tt arXiv:1610.0294}, for the case described there as Model B1, but generalizing it to allow for other possibilities for hierarchical charged fermion masses. 

Here we will focus solely on the quarks.  The Yukawa coupling matrices for the up and down quark sectors take the form
\begin{equation}
Y =  \frac{y_{u,d}}{\sqrt{3}} \left (\begin{array}{ccc} 1 & \beta_{1i} & \beta_{2i} \\ \beta_{1i} & 1 & \beta_{2i} \\ \beta_{3i} & \beta_{3i} & \beta_{4i}           \end{array} \right ),
\label{eq:yud}
\end{equation}
in which the index $i=u,d$, and we have suppressed the index on the label $Y$. Here the $y_{u,d}$ are dimensionless numbers that we include here to set an overall size for the $\beta$ coefficients in the up and down quark sectors, just for computational simplicity.  The messenger Yukawa matrices $Y^\prime_{1,2}$ are given by
\begin{equation}
Y^\prime_1=y_{u,d} \left (\begin{array}{ccc} -\frac{1}{2}-\frac{1}{2\sqrt{3}} & \frac{\beta_{1i}}{\sqrt{3}} & \;\; \frac{\beta_{2i}}{2} - \frac{\beta_{2i}}{2\sqrt{3}} \\  \frac{\beta_{1i}}{\sqrt{3}} & \;\; \frac{1}{2}-\frac{1}{2\sqrt{3}} & -\frac{\beta_{2i}}{2} - \frac{\beta_{2i}}{2\sqrt{3}} \\ \;\; \frac{\beta_{3i}}{2} - \frac{\beta_{3i}}{2\sqrt{3}} & -\frac{\beta_{3i}}{2} - \frac{\beta_{3i}}{2\sqrt{3}} & \frac{\beta_{4i}
}{\sqrt{3}}
\end{array} \right )
\end{equation}
\begin{equation}
\;\; Y^\prime_2=y_{u,d} \left (\begin{array}{ccc} \;\; \frac{1}{2}-\frac{1}{2\sqrt{3}} & \frac{\beta_{1i}}{\sqrt{3}} &  -\frac{\beta_{2i}}{2} - \frac{\beta_{2i}}{2\sqrt{3}} \\  \frac{\beta_{1i}}{\sqrt{3}} & -\frac{1}{2}-\frac{1}{2\sqrt{3}} & \;\; \frac{\beta_{2i}}{2} - \frac{\beta_{2i}}{2\sqrt{3}} \\ -\frac{\beta_{3i}}{2} - \frac{\beta_{3i}}{2\sqrt{3}} & \;\; \frac{\beta_{3i}}{2} - \frac{\beta_{3i}}{2\sqrt{3}} & \frac{\beta_{4i}
}{\sqrt{3}}
\end{array} \right ).
\end{equation}
In {\tt arXiv:1610.0294}, the $\beta$'s were all taken to be exactly one.  In this admittedly unrealistic situation, Eq.~(\ref{eq:yud}) has the highly symmetric form
\begin{equation}
Y = \frac{y_{u,d}}{\sqrt{3}} \left (\begin{array}{ccc} 1 &1 & 1 \\ 1 & 1 & 1 \\ 1 & 1 & 1           \end{array} \right ),
\end{equation}
which has two vanishing eigenvalues and one $O(1)$ eigenvalue.  The messenger Yukawa matrices both have a block diagonal form in the basis in which the quarks are diagonal. 

Let us now consider the case in which all of the $\beta$'s are real, but otherwise arbitrary. In our notation below, we will leave $\beta_{1,2,3,4}$ without an additional subscript to indicate $u$ or $d$, which is to be understood. With this in mind, the diagonalization of Eq.~(\ref{eq:yud}) takes place via the biunitary transformation,
\begin{equation}
U_L^\dagger Y U_R =Y_\text{diag},  
\end{equation}
in which
\begin{equation}
U_L^\dagger Y Y^\dagger U_L= (YY^\dagger)_\text{diag} =Y^2_\text{diag}, \qquad U_R^\dagger Y^\dagger Y U_R= (Y^\dagger Y)_\text{diag}= Y^2_\text{diag}.
\end{equation}
It is easily found that one of the eigenvalues of these Hermitian combinations is 
\begin{equation}
\lambda_1=\frac{y_{u,d}^2}{3}(-1+\beta_1)^2, 
\label{eq:evalsq1}
\end{equation}
with corresponding normalized eigenvector 
\begin{equation}
v_1=\frac{1}{\sqrt{2}}(1,-1,0).  
\end{equation}
The other eigenvalues are given by
\begin{equation}
\lambda_{2,3}= \frac{y_{u,d}^2}{6} \left ((1+\beta_1)^2+2(\beta_2^2+\beta_3^2)+ \beta_4^2 \mp \sqrt{\Lambda} \right ),
\label{eq:evalsqmp}
\end{equation}
in which $\Lambda$ is given by
\begin{equation}
\Lambda = (1+\beta_1)^4+4(\beta_2^4+\beta_3^4)+\beta_4^4+4((1+\beta_1)^2+\beta_4^2)(\beta_2^2+\beta_3^2)-2(1+\beta_1)^2\beta_4^2-8\beta_2^2\beta_3^2+16(1+\beta_1)\beta_2\beta_3 \beta_4.
\label{eq:Lambdadef}
\end{equation}
Note the $\beta_2\leftrightarrow \beta_3$ symmetry of Eqs.~(\ref{eq:evalsqmp})-(\ref{eq:Lambdadef}).  

 From the form of Eq.~(\ref{eq:evalsq1}) and Eq.~(\ref{eq:evalsqmp}), we can see that there are two general possibilities.  One option is that $\lambda_1$ is one of the small eigenvalues, which would have $\beta_1\rightarrow 1$, and $\lambda_2$ is the other, and hence $\lambda_3$ generically has an $O(1)$ value.  Another option is that $\lambda_1$ is the large eigenvalue, such that $\beta_1$ is appreciably different than 1, and both $\lambda_{2,3}$ are small.\\
 
\noindent $\bullet$ {\bf Case 1: $\lambda_{1,2}\ll \lambda_3$.}  This case, in which $\beta_1\rightarrow 1$, has been explored in {\tt arXiv:1812.10811} and {\tt arXiv:1912.12938}. For a second light eigenvalue, this case also requires $\beta_4=\beta_2\beta_3$.  In the limit that $\beta_{2,3}$ are taken to be large while the coefficients $y_{u,d}$ are simultaneously taken to be small so as to fix the physical third generation quark masses.  This is the $\mathcal{S}_3$ singlet-dominated regime, as $\beta_4$, which is the coupling of the $\mathcal{S}_3$ singlet quarks and Higgs-messenger field, is therefore much larger than all of the other Yukawa couplings.  This limit allows for the possibility of sizable stop mixing, and consequently relatively light superpartners ($5-6$ TeV) compared to other viable limits of the parameter space.  We will not discuss this case further in these notes. \\





\noindent $\bullet$ {\bf Case 2: $\lambda_{2,3}\ll \lambda_1$.}  This is the case we will focus on in these notes.  Here it is necessary that $\beta_1\neq 1$ such that $\lambda_1\gg \lambda_{2,3}$.  We also see from the form of Eqs.~(\ref{eq:evalsqmp})-(\ref{eq:Lambdadef}) that this case would require $\beta_1\rightarrow -1$ and $\beta_{i=2,3,4}\ll 1$, as well as $\Lambda \rightarrow 0$.  Indeed, $\lambda_{2,3}=0$ is achieved for $\beta_1=-1,\beta_2=\beta_3=\beta_4=0$.  
For $\beta_1=-1$ only, the condition for $\Lambda=0$ is as follows:
\begin{equation}
-8 \beta_2^2 \beta_3^2+4(\beta_2^4+\beta_3^4)+4(\beta_2^2+\beta_3^2)\beta_4^2+\beta_4^2=0,
\end{equation}
which is zero only for $\beta_4=0$, $\beta_2=\beta_3$.
Let us now consider this specific case but leave $\beta_2$ and $\beta_3$ unconstrained at present, recalling that eventually we will need to restrict ourselves to the case that $\beta_2\approx \beta_3$, and $\beta_{2,3}\ll 1$.  In this case, the mass eigenvalues thus take the form
\begin{equation}
U_L^\dagger Y U_R= Y_\text{diag}=\frac{y_u}{\sqrt{3}}\text{Diag}(\sqrt{2}\beta_2, \sqrt{2}\beta_3,2),
\end{equation}
which shows that we should take $y_t =y_u/(2/\sqrt{3})$ in this example to identify $y_t$ with the top quark Yukawa coupling.  Furthermore, the diagonalization matrices $U_{L,R}$ take the form
\begin{equation}
U_L=\left (\begin{array}{ccc} \frac{1}{\sqrt{2}} & 0 & \frac{1}{\sqrt{2}} \\ \frac{1}{\sqrt{2}} & 0 & -\frac{1}{\sqrt{2}} \\ 0& 1 & 0 
\end{array} \right )
\end{equation}
\begin{equation}
U_R=\left (\begin{array}{ccc} 0& \frac{1}{\sqrt{2}} &  \frac{1}{\sqrt{2}} \\ 0 & \frac{1}{\sqrt{2}} &  -\frac{1}{\sqrt{2}} \\ 1 & 0 & 0 
\end{array} \right ).
\end{equation}
The messenger Yukawas in this basis take the form:
\begin{equation}
Y_1^\prime = y_t \left (\begin{array}{ccc} -\frac{\beta_2}{2\sqrt{2}}& -\frac{3}{4} & -\frac{\sqrt{3}}{4} \\ \;\;\; 0 &  -\frac{\beta_3}{2\sqrt{2}} & \frac{\beta_3}{2}\sqrt{\frac{3}{2}}\\ \frac{\beta_2}{2}\sqrt{\frac{3}{2}} & -\frac{\sqrt{3}}{4} & \;\;\; \frac{1}{4} \end{array} \right ),\qquad 
%\end{equation}
%\begin{equation}
Y_2^\prime = y_t\left (\begin{array}{ccc}  -\frac{\beta_2}{2\sqrt{2}}& -\frac{3}{4} & \frac{\sqrt{3}}{4} \\ \;\;\; 0 &  -\frac{\beta_3}{2\sqrt{2}} & -\frac{\beta_3}{2}\sqrt{\frac{3}{2}}\\ -\frac{\beta_2}{2}\sqrt{\frac{3}{2}} & \frac{\sqrt{3}}{4} & \;\;\; \frac{1}{4} \end{array} \right )
\end{equation}
Let us assume that there are the same type of structures for the down-type quarks and for the charged leptons. 

We turn now to the flavored gauge mediation contribution to the soft supersymmetry breaking parameters of the scalar sector.  Anticipating that (with the mass ordering given above) the $\beta_2$ parameters are proportional to first generation fermion masses, we will neglect them to leading order, and furthermore only consider leading order (linear) corrections in the second generation $\beta_3$ parameters. In what follows, all soft scalar mass-squared parameters are assumed to include a factor of $\Lambda^2/(4\pi)^4$, all trilinear scalar couplings are assumed to include a factor of $\Lambda/(4\pi)^2$, and we define the following combinations of the gauge couplings:  
\begin{equation}
g_{\tilde{u}}^2 =\frac{13}{30} g_1^2+\frac{3}{2}g_2^2+\frac{8}{3}g_3^2, \qquad g_{\tilde{d}}^2 =\frac{7}{30} g_1^2+\frac{3}{2}g_2^2+\frac{8}{3}g_3^2, \qquad g_{\tilde{l}}^2 =\frac{9}{10} g_1^2+\frac{3}{2}g_2^2.
\end{equation}
The nonvanishing corrections to the soft supersymmetry breaking terms then take the form:
\begin{eqnarray}
(\delta m_{\tilde{Q}}^2)_{11}&=&\frac{195}{16}(y_t^4+y_b^4)+\frac{15}{4}y_t^2 y_b^2+\frac{27}{16}y_b^2 y_\tau^2 - 3y_t^2 g_{\tilde{u}}^2-3y_b^2 g_{\tilde{d}}^2, \nonumber \\
(\delta m_{\tilde{Q}}^2)_{33}&=&\frac{39}{16}(y_t^4+y_b^4)+\frac{5}{4}y_t^2 y_b^2+\frac{11}{16}y_b^2 y_\tau^2 - y_t^2 g_{\tilde{u}}^2-y_b^2 g_{\tilde{d}}^2, \nonumber \\
(\delta m_{\tilde{Q}}^2)_{12}&=&(\delta m_{\tilde{Q}}^2)_{21}=-\frac{3}{4\sqrt{2}}(y_t^4\beta_{3u}+y_b^4 \beta_{3d}-y_t^2 y_b^2 (\beta_{3u}+\beta_{3d})), \nonumber \\
%\end{eqnarray}
%\begin{eqnarray}
(\delta m_{\tilde{u}}^2)_{22}&=&\frac{189}{8}y_t^4+\frac{9}{2}y_t^2 y_b^2 - 6y_t^2 g_{\tilde{u}}^2, \qquad 
% \nonumber \\
(\delta m_{\tilde{u}}^2)_{33}=
%&=&
\frac{45}{8}y_t^4+\frac{1}{2}y_t^2 y_b^2- 2y_t^2 g_{\tilde{u}}^2, \nonumber \\
(\delta m_{\tilde{d}}^2)_{22}&=&\frac{189}{8}y_b^4+\frac{9}{2}y_t^2 y_b^2+\frac{27}{8}y_b^2y_\tau^2 - 6y_b^2 g_{\tilde{d}}^2 \nonumber \\
(\delta m_{\tilde{d}}^2)_{33}&=&
\frac{45}{8}y_b^4+\frac{1}{2}y_t^2 y_b^2+\frac{11}{8}y_b^2y_\tau^2- 2y_b^2 g_{\tilde{d}}^2,
\end{eqnarray}
\begin{eqnarray}
(\delta m_{\tilde{L}}^2)_{11}&=&\frac{141}{16}y_\tau^4+\frac{81}{16}y_b^2 y_\tau^2 - 3y_\tau^2 g_{\tilde{l}}^2, \qquad
% \nonumber \\
(\delta m_{\tilde{L}}^2)_{33}=
%&=&
\frac{17}{16}y_\tau^4+\frac{33}{16}y_b^2 y_\tau^2 - y_\tau^2 g_{\tilde{l}}^2,  \nonumber \\
(\delta m_{\tilde{L}}^2)_{12} &=&
(\delta m_{\tilde{L}}^2)_{21}=-\frac{3}{4\sqrt{2}}(y_\tau^4\beta_{3l}), \nonumber \\
%\end{eqnarray}
%\begin{eqnarray}
(\delta m_{\tilde{e}}^2)_{22}&=&\frac{135}{8}y_\tau^4+\frac{81}{8}y_b^2y_\tau^2 - 6y_\tau^2 g_{\tilde{l}}^2, \qquad 
% \nonumber \\
(\delta m_{\tilde{e}}^2)_{33}=
%&=&
\frac{23}{8}y_\tau^4+\frac{33}{8}y_b^2 y_\tau^2- 2y_\tau^2 g_{\tilde{l}}^2,
\end{eqnarray}
\begin{eqnarray}
\delta m_{h_u}^2=-\frac{9}{2}y_t^4-\frac{3}{2}y_t^2y_b^2,\qquad \delta m_{h_d}^2=-\frac{9}{2}y_b^4-\frac{3}{2}y_\tau^4-\frac{3}{2}y_t^2y_b^2,
\end{eqnarray}
\begin{eqnarray}
(\tilde{A}_u)_{22}&=&-\frac{3}{\sqrt{2}}y_t^3\beta_{3u}, \qquad (\tilde{A}_u)_{33}=-\frac{3}{2}y_t^3-\frac{1}{2}y_t y_b^2, \nonumber \\
(\tilde{A}_d)_{22}&=&-\frac{3}{\sqrt{2}}y_b^3\beta_{3d}, \qquad (\tilde{A}_d)_{33}=-\frac{3}{2}y_b^3-\frac{1}{2}y_b y_t^2, \nonumber \\
(\tilde{A}_e)_{22}&=&-\frac{3}{\sqrt{2}}y_\tau^3\beta_{3l}, \qquad (\tilde{A}_e)_{33}=-\frac{3}{2}y_\tau^3. 
\end{eqnarray}
If the second generation masses are set to zero $(\beta_{3u}=\beta_{3d}=\beta_{3l}=0)$, all soft mass-squared parameters are flavor diagonal, and all trilinear scalar couplings have only $33$ entries.
\end{document}

In the limit that $\beta_{2,3}\ll 1$, $\delta m_{\tilde{Q}}^2$, $\delta m_{\tilde{u}}^2$, and $\tilde{A}_u$ take the form (neglecting terms of $O(\beta_2^2)$, $O(\beta_3^2)$, and $O(\beta_2\beta_3)$):
\begin{equation}
\delta m_{\tilde{Q}}^2=\frac{\Lambda^2}{(4\pi)^4} \left (\begin{array}{ccc} 
\frac{195}{16} y_u^4-\left (8 g_3^2+\frac{9}{2} g_2^2+\frac{13}{10}g_1^2 \right ) y_u^2&-\frac{3}{4\sqrt{2}}\beta_3 y_u^4 &0\\ -\frac{3}{4\sqrt{2}}\beta_3 y_u^4&0  & 0\\ 0 & 0 & 
\frac{39}{16} y_u^4-\left (\frac{8}{3} g_3^2+\frac{3}{2} g_2^2+\frac{13}{30}g_1^2 \right )
\end{array} \right )\nonumber
\end{equation}
\begin{equation}
\delta m_{\tilde{u}}^2=\frac{\Lambda^2}{(4\pi)^4} \left (\begin{array}{ccc} 0 & 0 & 0 \\ 0 & 
\frac{189}{8} y_u^4-\left (16 g_3^2+9g_2^2+\frac{13}{5}g_1^2 \right ) y_u^2 &0 \\ 0& 0 &  \frac{45}{8} y_u^4-\left (\frac{16}{3} g_3^2+3 g_2^2+\frac{13}{15}g_1^2 \right ) y_u^2
\end{array} \right )\nonumber
\end{equation}
\begin{equation}
\tilde{A}_u=-\frac{\Lambda}{(4\pi)^2}\left (\begin{array}{ccc} \frac{3}{2\sqrt{2}}\beta_2y_u^3& 0 & 0\\ 0&\frac{3}{\sqrt{2}}\beta_3 y_u^3&0 \\ 0& 0& \frac{3}{2} y_u^3\end{array} \right )\nonumber
\end{equation}
\begin{equation}
\delta m_{H_u}^2 = -\frac{\Lambda^2}{(4\pi)^4}  \frac{9}{2} y_u^4.\nonumber
\end{equation}

The soft terms involve rather lengthy but straightforward expressions, as seen below.  The nonzero terms are:
\begin{eqnarray}
\delta m_{\tilde{Q}_{11}}^2&=&\frac{65}{3}-\frac{26}{15}g_1^2-6 g_2^2-\frac{32}{3}g_3^2+\beta_2^2 \left (\frac{38}{3}-\frac{13}{15}g_1^2-3g_2^2-\frac{16}{3}g_3^2+\frac{23}{3}\beta_2^2\right )\nonumber \\ &+&\beta_3^2 \left (\frac{16}{3}+\frac{16}{3}\beta_2^2+\frac{4}{3}\beta_2^4 \right ) \nonumber \\
\delta m_{\tilde{Q}_{22}}^2&=& \frac{\beta_3^2}{5(2+\beta_3^2)} \Big [215-26 g_1^2-90g_2^2+190 \beta_3^2+\beta_2^4(65+70\beta_3^2) \nonumber \\ &+&\beta_2^2(170-13 g_1^2-45 g_2^2-80 g_3^2+100 \beta_3^2) \Big ] \nonumber 
\end{eqnarray}
\begin{eqnarray}
\delta m_{\tilde{Q}_{23}}^2&=&\delta m_{\tilde{Q}_{32}}^2= \frac{\sqrt{2}\beta_3(\beta_3^2-1)}{15(2+\beta_3^2)} \Big [ 245-26 g_1^2-90 g_2^2-160 g_3^2+205 \beta_3^2 \nonumber \\ &+&\beta_2^2(110-13 g_1^2-45 g_2^2-80 g_3^2+70 \beta_3^2)+\beta_2^4 (95+85 \beta_3^2)\Big] \nonumber \\
\delta m_{\tilde{Q}_{33}}^2&=&\frac{1}{2+\beta_3^2}\Big[2 \beta_2^4(1-6\beta_3^2-3\beta_3^4+2\beta_3^6)+\frac{2}{45} \Big (195-450 \beta_3^2-225\beta_3^4+210 \beta_3^6\nonumber \\ &-&(1-\beta_3^2)^2(2+\beta_2^2)(13 g_1^2+45 g_2^2+80 g_3^2) + 30 \beta_2^2(7+6\beta_3^2+3\beta_3^4+2\beta_3^6) \Big)\nonumber
\end{eqnarray}